\documentclass[12pt]{report}

\usepackage[T1]{fontenc}
\usepackage[utf8]{inputenc}
\usepackage{graphicx}

\begin{document}
 
{\Huge Optoacopladores y relevadores}\\
\includegraphics[scale=1]{upzmg.jpg} \\
{\Huge Enesto Alonso Partida López\\ Universidad Politecnica De La Zona Metropolitana De Guadalajara\\ Mecatronica 4 A\\ Septiembre-diciembre 2019}
\date{3 de octubre 2019}
 
\newpage

{\huge \textbf{INTRODUCCION:}\\}\\


{\large Optoacoplador conocido como optoaislador o aislador acoplado opticamente que
es dispositivo de emision y recepcion que funciona como interruptor activos mediante
luz emitida por diodo led que satura al componente optoelectronico.
relevador dispositivo electromagnetico que funciona como interruptor controlado
por el circuito controlado electronicamente, esto por medio de bobina y
electroiman, este enciende un juego que permite abrir o cerrar otros circuitos
electronicosindependientes, el PLC son utilizados en la industria diseñados para
multiples señales de entrada y salidas, rangos de temperatura ampliados, inmunidad
al ruido electrico, resistentes a la vibracion y al impacto y estos vienen
siendo sistemas de tiempo real..}\\
 

{\huge \textbf{OBJETIVO}\\}\\


{\large Armado del pototipado de una PLC, con placas de arduino con cuatro placas de
esta misma para observar el funcionamiento real de un PLC registrando salidas
de voltaje de 5 V. y 12 V. en cada salida de nuestro circuito.}\\



{\huge \textbf{MARCO TEORICO}\\}\\


{\large PLC es componente basico en el mundo de automatizaciones industriales. las
aplicaciones industriales hizo que el sistema PLC fueran costosos, para el comprar
o reparar, arduino es especie de controlador programable universal, aunque
es solo nucleo en cualquier cosa, en todo caso construyendo aplicaciones generales,
con un hardware externo, especialmente las interfaces capaces de transferir
señales a sensores con direccion a los actuadores asi reduciendo la EMI que 
dañe el microcontrolador o el software\\}\\


{\huge \textbf{MATERIALES}\\}\\


\begin{enumerate}
\item Protoboard
\item Placa de arduino
\item Relevadores
\item Resistencias
\item Optoacopladores
\item 2N2222
\item Fuente de Voltaje
\item Leds
\item Push Botton
\item Clabe Para Protoboard
\item Caimanes
\end{enumerate}


{\huge \textbf{Desarrollo}\\}\\
{\large Procedimos armar el circuito como se muestra en la imangen 1. que se encuentra
en la carpeta de este documento,
como observamos ahi dos resistencias sin saber su valor es varioable debido
al L1823, que este corriente soporte, procedimos a aplicar la ley de ohm que
es R = V/I, la corrientees la cual es 80mA. nuestro voltaje es el marcado en
el circuito, el resultado fue de 150 ohms. por desicion decidimos colocar una
resistencia de 220 ohms. para evitar sobre calentamiento en la resistencia o que
dañe alguno de nuestros componentes.
despues procedimos a medir nuestro 2N2222A ya que este cambia su valor
de HFE, debido para como muestra el datashep ya conociendo este resultado
medido procedemos a calcular el valor de nuestra resistencia con la siguiente
operacion R = (Vin - 0.6)(HFE)/ A. el resultado de nuestra operacion fue de
3950 ohms lo cual equivale a una resistencia de 4 k. y todo esto va conectado a
la tierra y este paso a las entradas de la bobinas posteriormente se conectaron
en paralelo dos diodos y una resistencia la cual este fue un led para verificar la
cantidad luminosa que este generaba. procedimos a colocar los cables machos
al aire para esperar el codigo de compilacion que tendrianuestro arduino, para
la verifacacion de este en la interfaz de entrada se tiene que registra 12 V. en
la salida de 5 V. y que nuestro relevador haga un pequeño sonido de click y
con esto sabremos que todo quedo bien colocado en nuestro protoboard, con
quede el resultado de estos voltajes de 12 V. y 5 V. seria nuestro resultados que
esperamos obtener}\\

{\huge \textbf{Resultados obtenidos}\\}\\
{\large Esta fue nuestra primera operacion para calcular el primer valor de nuestra
resistencia que viene despues del push button:\\
R = 12 V/ 80mA = 150 ohm\\
la segunda operacion para calcular nuestra segunda resistencia que esta colocada
en la salida de nuestro arduino utilizamos la siguiente operacion:\\
R = (Vin - 0.6) (HFE)/12 A.\\
El HFE lo obtuvimos mediante la medidcion del 2N2222A. en un multimetro
con entrada para la medicion de transistores. asi fue como conocimos el valor
de nuestro 2N2222A.\\
el cual fue nuestra operacion: R= (12 v - 0.6)(250 HFE) / 12 A= 3950}\\
{\huge \textbf{conclusion}\\}\\
{\large En la practica previamente realizada, aprendidos el funcionamiento de los optoacopladores
el cual es un circuito integrado por un led el cual acciona el paso
de corriente del otro extremo, y los relevadores es un componente que esta integrado
por una bobina que acciona un movimiento el cual perimte el paso de
la corriente. ademas se logro el cometidodespues de muchos intentos el cual era
que mediante una programacion en una placa de arduino, esta mandara la señal de cuando un boton permitia el paso de corriente por todo el circuito, lo mas
dificil fue la programacion ya que habia muchas maneras de realizar mas sin embargo
no todos el circuito, lo mas dificil fue la programcion funcionaron, otros
problemas fueron las resistencias de los valores poco comunes, pero como se dijo
anteriormente despues de varios intentos se logro realizar en pocas palabras un
PLC casero completamente funcional, el cual marcaria cuando esta funcionando
correctamente.}
\newpage
{\huge \textbf{Bibliografia:}\\}\\
@online{Electronica Unicrom,
author = {Luis González López},
title = {Amplificadores de Potencia: clasificación, clase A, B, AB, C},
year = {2016},
url = {https://unicrom.com/amplificadores-de-potencia-clasificacion/},
OPTsubtitle = {Amplificadores clase A},
OPTlanguage = {Español},
OPTversion = {1},
OPTdate = {21},
OPTmonth = {6},
OPTurldate = {https://unicrom.com/amplificadores-de-potencia-clasificacion/},
}


\end{document}